% --------------------------------------------------------------
% This is all preamble stuff that you don't have to worry about.
% Head down to where it says "Start here"
% --------------------------------------------------------------
 
\documentclass[12pt]{article}
 
\usepackage[margin=1in]{geometry} 
\usepackage{amsmath,amsthm,amssymb}
 
\newcommand{\N}{\mathbb{N}}
\newcommand{\Z}{\mathbb{Z}}
 
\newenvironment{theorem}[2][Theorem]{\begin{trivlist}
\item[\hskip \labelsep {\bfseries #1}\hskip \labelsep {\bfseries #2.}]}{\end{trivlist}}
\newenvironment{lemma}[2][Lemma]{\begin{trivlist}
\item[\hskip \labelsep {\bfseries #1}\hskip \labelsep {\bfseries #2.}]}{\end{trivlist}}
\newenvironment{exercise}[2][Exercise]{\begin{trivlist}
\item[\hskip \labelsep {\bfseries #1}\hskip \labelsep {\bfseries #2.}]}{\end{trivlist}}
\newenvironment{reflection}[2][Reflection]{\begin{trivlist}
\item[\hskip \labelsep {\bfseries #1}\hskip \labelsep {\bfseries #2.}]}{\end{trivlist}}
\newenvironment{proposition}[2][Proposition]{\begin{trivlist}
\item[\hskip \labelsep {\bfseries #1}\hskip \labelsep {\bfseries #2.}]}{\end{trivlist}}
\newenvironment{corollary}[2][Corollary]{\begin{trivlist}
\item[\hskip \labelsep {\bfseries #1}\hskip \labelsep {\bfseries #2.}]}{\end{trivlist}}
 
\begin{document}
 
% --------------------------------------------------------------
%                         Start here
% --------------------------------------------------------------
 
%\renewcommand{\qedsymbol}{\filledbox}
 
\title{Homework 1}%replace X with the appropriate number
\author{Cory Rindlisbacher\\ %replace with your name
} %if necessary, replace with your course title
 
\maketitle
 
\begin{exercise}{1} %You can use theorem, proposition, exercise, or reflection here.  Modify x.yz to be whatever number you are proving
It is believed that space has some “graininess” characterized by a length L, although this has never been measured. However, the following parameters which have been measured are believed to be relevant: $c = 3 \times 10^{− 8} m s^{− 1}$, the speed of light,$\hslash = 6.62 \times 10^{− 34} kg m^{2} s^{− 1}$, the Planck constant, and $G = 6.67 \times 10^{− 11} m kg^{− 1} s^{− 2}$, the gravitational constant. Using only these three constants, find a ballpark estimate for L. Compare with the radius of an atom.

\end{exercise}
 
\begin{exercise}{2}
Consider a pendulum. Let $T$ be the period of the oscillation, $d$ be the maximum displacement of the pendulum to vertical, $l$ be the length of the pendulum, $m$ be the mass of the weight, and $g$ be the acceleration due to gravity. Assuming these are the only relevant parameters, what is the form of the relation you expect to see between them?
\end{exercise}

\begin{exercise}{3}
A farmer wants to build a windmill to generate power. Assuming that the power $P$
generated is a function the density of the air $\rho$, viscosity $\mu$, diameter of the windmill $d$, wind speed $v$, and rotational speed $\omega$, can you express $P$ in terms of an unknown function of as few variables as possible? Can you suggest anything about the form of this function? If you were to build a scale model of the windmill, how would you do it and what correspondence would you make between the power generated by the scale model and the expected performance of the real windmill?
\end{exercise}

\begin{exercise}{4}
At a paper mill, paper is wound onto a cylinder. Suppose there is a counter which counts the number of revolutions of the cylinder. Find the appropriate relation between the length of paper wound on the cylinder and the number of revolutions. [Hint: it’s non-linear.] Comment on sources of error.
\end{exercise}

% --------------------------------------------------------------
%     You don't have to mess with anything below this line.
% --------------------------------------------------------------
 
\end{document}